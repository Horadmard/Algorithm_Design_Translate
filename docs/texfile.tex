% !TEX TS-program = xelatex
% !TEX encoding = UTF-8

% This is a simple template for a XeLaTeX document using the "article" class,
% with the fontspec package to easily select fonts.

\documentclass{book} % use larger type; default would be 10pt

\usepackage{xepersian} 

\settextfont{Yas}
\setlatintextfont{Bitstream Charter}
\setdigitfont{Yas}

\begin{document}
%\setsansfont{Deja Vu Sans}
%\setmonofont{Deja Vu Mono}

\title{\lr{Algorithm Design: 
A Methodological Approach
Translation}}
\author{Hossein Radmard}
% \date{}
%\date{} % Activate to display a given date or no date (if empty),
         % otherwise the current date is printed 

\maketitle

\section{مسئله 83 رنگ کردن یک گراف دو رنگی}

در این مسئله، مانند موارد بعدی، حل مسئله هدف ماست و بهینه بودن جواب برای ما حائز اهمیت نیست. دراینجا، ما الگوریتمی را برای رنگ کردن گراف با دردست داشتن تنها دو رنگ، بررسی میکنیم. نسخه‌ی کلی‌تر(رنگ آمیزی با هر تعداد رنگ) در مسئله‌ی 59، صفحه‌ی 236 بررسی میشود. گرچه، نسخه‌ی فعلی بسیار بهینه‌تر است. در ادامه، این بحث بسیار به موضوع "گرافهای پرکاربرد: گراف‌های دوبخشی" مرتبط است.

گراف همبند بدون جهت 
{G = N, V)}\lr
 (ناتهی) به ما داده شده‌ست. ما قصد داریم با رنگ‌های سیاه و سفید گراف را رنگ آمیزی کنیم به گونه‌ای که هیچ دو راس همسایه‌ای دارای رنگ یکسانی نباشند. چنین گرافی را گراف دورنگی مینامیم. الگوریتم حریصانه ای که ما قصد ساخت آن را برای این منظور داریم به پیمایش سطری گراف ها مرتبط است که آن را در ابتدا بررسی کردیم.


 چنین گرافی را گراف دورنگی مینامیم. الگوریتم حریصانه ای که ما قصد ساخت آن را برای این منظور داریم به پیمایش سطری گراف ها مرتبط است که آن را در ابتدا بررسی کردیم.
\subsection*{پیمایش سطری گراف: یادآوری}
\subsubsection*{معرفی}

اول از همه، اجازه دهید مفاهیم "فاصله‌ی میان دو راس" و "پیمایش سطری" را برای گراف‌های همبندِ بدون جهت تعریف کنیم.

تعریف 10 (فاصله‌ی میان دو راس):
در نظر میگیریم، G = (N, V) یک گراف همبند بدون جهت و s و s' دو راس این گراف باشند. طول کوتاه‌ترین مسیر میان s و s' را فاصله‌ی میان s و s' گویند.


تعریف 11 (جستجوی سطری):
فرض کنیم G یک گراف همبند و بدون جهت و s یکی از راس های آن باشد. هر فرایندی که با افزایش فاصله‌ها از راس s با راس های گراف G برخورد میکند به عنوان پیمایش سطری گراف G از s شناخته میشود.  



از نمودار (b) در شکل 7.8 صفحه‌ی 363، میتوانیم نتیجه بگیریم که لیست ⟨a, b, c, d, e, f, g, h⟩ با پیمایش سطری با شروع از راس a مطابقت دارد. و همین مطلب برای لیست ⟨a, c, b, d, e, h, f, g⟩ نیز صدق میکند.

تصویر 7.8 - یک مثال از گراف. تصویر (a) گرافی را نمایش میدهد که مثالی از حالت مسئله را نشان میدهد.  تصویر (b) کوتاه‌ترین مسیر راس a را تا هر راس گراف با خطوط پررنگ نشان میدهد. در تصویر (b)، عددی که در هر راس مشخص است در واقع فاصله‌ی آن راس تا راس a است.

%\subsection{}

\subsubsection*{حلقه بدون تغییر}

ما علاقه داریم یک الگوریتم بدون تغییر بسازیم؛ یک الگوریتم حریصانه؛ و اینگونه خود را محدود میکنیم. برای جستجوی حلقه بدون تغییر، ادامه ی این ساز و کار به خواننده واگذار میشود. اکنون تصور میکنیم قسمتی از کار انجام شده ست(بخش ۳، صفحه ۹۳ را ببنید). به این ترتیب، برای یک گراف جزئی G′ = (N′, V′) (زیرگراف G القا شده با مجموعه رئوس 'N، شامل رئوس ابتدایی)، لیستی تشکیل شده از پیمایش سطری 'G با شروع از s داریم.
عموما* این لیست، CLOSE نامیده میشود. پیشرفت این روند شامل گستردن این لیست با افزودن رئوسی است که در CLOSE نیستند و تا جای ممکن به s نزدیکند.

\end{document}
